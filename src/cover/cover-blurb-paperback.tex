\documentclass[
    coverheight=9.25in,
    coverwidth=6.125in, % (pagesize - spinewidth) / 2
    spinewidth=1.125in,
    bleedwidth=0in,
    11pt,
    marklength=0pt,
  ]{bookcover}
  
  \usepackage{lettrine}
  \usepackage{fancybox}
  \usepackage{wrapfig}
  \usepackage[many]{tcolorbox}
  \usetikzlibrary{calc,positioning, shadings}
  \usepackage[T1]{fontenc}
  \usepackage{Alegreya} %% Option 'black' gives heavier bold face 

  \newcommand{\olpath}{../}
  \newcommand{\whitebg}[1]{%
  \tikz\node[circle,draw,minimum size=1.1cm,
  fill=white,
  path picture={
      \node at (path picture bounding box){
          \includegraphics[width=1.1cm]{\olpath#1}
      };
  }]{};
  }
  \newcommand{\bartosz}{
    \vspace{0pt}
    \begin{tcolorbox}[beamer,
      width=3.6cm,
      arc=0pt,
      boxsep=0pt,
      left=0pt,right=0pt,top=0pt,bottom=0pt,
      ] \includegraphics[width=3.6cm]{bartosz}
    \end{tcolorbox}
  }
  \input{\olpath/version}
  
  \definecolor{BackgroundColor}{HTML}{f3f6ed}
  \definecolor{SpineBackColor}{HTML}{262626}
  \definecolor{SpineFontColor}{RGB}{248,154,14}
  
  \newcommand{\stripskip}{4}
  \newcommand{\stripwidth}{3}

  \begin{document}
  
  \begin{bookcover}
    \bookcovercomponent{color}{bg whole}{color=BackgroundColor}
    \bookcovercomponent{color}{spine}{color=SpineBackColor}
    \bookcovercomponent{normal}{front}{
      \begin{tikzpicture}[
        overlay,
        remember picture,
        ribbon/.style={anchor=center, rotate = 45,
                             font={\fontsize{20}{1}\selectfont\bfseries\scshape}}
                        ]
        \coordinate (A) at ($ (current page.south east) + (-\stripskip,0) $);% <-- changed coordinate from 'north' to south'
        \coordinate (A') at ($(A) + (-\stripwidth,0) $);

        \coordinate (B) at ($ (current page.south east) + (0,\stripskip) $);% <-- changed coordinate from 'north' to south' and sign for \stripskip
        \coordinate (B') at ($(B) + (0,\stripwidth) $);% <-- changed sign for \stripskip 

        \fill [red!20] (A) -- (A') -- (B') -- (B) -- cycle;

        \coordinate (tempA) at ($(A)!.5!(A')$);
        \coordinate (tempB) at ($(B)!.5!(B')$);

        \node [ribbon](text) at ($(tempA)!.5!(tempB)$) {
          \raisebox{-.35\height}{\includegraphics[width=.8cm]{\olpath/fig/icons/scala}}
          Scala Edition
        };

    \end{tikzpicture}
    
    \begin{center}
      \fontsize{40pt}{7em}\selectfont\bfseries
          CATEGORY THEORY \\FOR PROGRAMMERS
      \vfil
      % \vspace*{1cm}
      \includegraphics[width=.5\coverwidth]{piggie}
      \vfil
      \normalfont\Huge
      \textbf{Bartosz Milewski}
      \vfil
      \vspace*{1cm}
    \end{center}}
    
    \bookcovercomponent{center}{spine}{
      \rotatebox[origin=c]{-90}{\color{SpineFontColor}
      \bfseries\Huge Category Theory for Programmers \hspace{2em} Bartosz Milewski}}

    \bookcovercomponent{normal}{back}{%
    \begin{minipage}[b][\coverheight][t]{\coverwidth}
      \begin{center}
        \includegraphics[width=.8\coverwidth]{bunnies}
        \begin{minipage}[t]{.8\coverwidth}
          \input{blurb}
          \vspace{.5cm}
        \end{minipage}
        
        \begin{minipage}{.85\textwidth}
          \rule{\textwidth}{0.4pt}

          \begin{tabular}[h]{p{3.4cm} p{\textwidth}}
            \bartosz
            &
            \vspace{5pt}
            \begin{minipage}[b]{.58\coverwidth}
              \fontsize{11pt}{1.4em}\selectfont\textit{Category Theory for Programmers}
                is a series of blog posts by Bartosz Milewski, originally at bartoszmilewski.com.\\
                Edited by Igal Tabachnik. Licenced under CC BY-SA 4.0.
          \end{minipage}
        \end{tabular}
        \begin{flushright}
          \vspace{-2.6cm}
          \begin{minipage}[b]{4cm}
          \raggedleft
          \whitebg{fig/icons/by}
          \whitebg{fig/icons/cc}
          \whitebg{fig/icons/sa}
          \centering\footnotesize{\texttt{\OPTversion}}
          \end{minipage}
        \end{flushright}
      \end{minipage}
    \end{center}
  \end{minipage}
  }
  \end{bookcover}
\end{document}